%% LyX 2.1.5dev created this file.  For more info, see http://www.lyx.org/.
%% Do not edit unless you really know what you are doing.
\documentclass[english]{article}
\usepackage[T1]{fontenc}
\usepackage[latin9]{inputenc}
\usepackage{geometry}
\geometry{verbose,tmargin=2cm,bmargin=2cm,lmargin=2cm,rmargin=2cm,headheight=2cm,headsep=2cm,footskip=2cm}
\usepackage{float}
\usepackage{babel}
\begin{document}

\title{Additively Manufactured Engine Lowering Inefficiency at Altitude
(AMELIA) - Project Plan}


\author{Peter Senior}


\date{2015-09-xx}

\maketitle

\section{Project Scope}

This project aims to design, build and test a ramjet combustor and
nozzle capable of producing a net thrust at its design conditions.
An extension is to ensure the engine operates well at off-design conditions,
for example higher or lower speeds and atmospheric pressures. The
supply nozzle and intake will be designed in a separate project, but
the two projects will work together closely.

To allow the use of complex shapes in the design and to reduce manufacturing
costs in the long term the engine will be constructed primarily through
additive manufacturing, also known as 3D-Printing. 3D Printing is
also more environmentally conscious than subtractive manufacturing
(such as the use of a lathe or chisel), as it substantially reduces
waste produced. When a sufficiently capable machine is used, final
manufactured items can be far more intricate and complex than items
made by hand.

Computational Fluid Dynamics will be used to design and optimise the
engine before manufacture, in order to increase the performance achieved
without having to iterate on a physical design. 

In order to carry out what is a highly ambitious project the design
process will be arranged so that a viable solution is produced as
soon as possible. This solution will then be matured over time to
give increased performance before a decision is made after the interim
report whether to go ahead with construction. The project will be
structured around the CADMID (Concept, Assessment, Demonstration,
Manufacture, In-Service, Disposal) cycle which is standard for UK
Ministry of Defence (MoD) projects. A brief overview of the CADMID
cycle is contained in Appendix \ref{sec:The-CADMID-Cycle}. The CADMID
cycle will be used because it has been demonstrated by the MoD to
be effective for the management of projects with a single final manufactured
solution.

It is envisaged that time spent maturing the design will focus on
the combustion chamber, as this is considered the highest risk component.
This is because the majority of work done advancing combustion chamber
technology is highly proprietary in nature. Basic combustion chambers
are however easy to design, however they are often extremely inefficient.


\section{Major Risks}

There are several major hurdles to be overcome during the life of
this project:
\begin{itemize}
\item Personal training:

\begin{itemize}
\item This project will leverage Computational Fluid Dynamics (CFD) a great
deal. Accurate and reliable use of CFD requires significant levels
of training, alongside a good level of understanding of the underlying
theory. Effective use of CFD is as much an art form as it is a science.
It must be ensured that personal training is at a sufficient level
that major mistakes are not made early in the project which are costly
to rectify later on.
\item If this project goes to manufacture it will utilise two pressurised
gas systems (pressurised air and a pressurised gas fuel supply). It
is vital from a safety perspective that training in the use of these
systems has been completed.
\item This project will involve the installation and running of the engine
in the engine test cells, this will require training in the engine
test cell procedures. Training may also be required in workshop use
and possibly welding.
\end{itemize}
\item Safety hazards:

\begin{itemize}
\item This project may use a high-voltage spark for ignition. This is a
safety hazard and must be taken into account with regards to risk
assessment and training.
\item This project will use combustible materials in an untested system.
There is therefore a significant risk of fire, and a slight risk of
explosion. This must be taken into account at all stages of the project,
particularly so that the system fails safely.
\end{itemize}
\item Design process hazards:

\begin{itemize}
\item If a workable solution cannot be created in the time constraints,
or the project sponsor is not sufficiently satisfied with the design,
there is the option to extend the demonstration phase past the interim
design review, and to push back the manufacturing date. This would
have the knock-on effect of curtailing (or removing altogether) testing
time. If the design is still not sufficiently mature then the physical
construction of the engine may not occur at all. Obviously this is
not a preferred option.
\end{itemize}
\end{itemize}

\section{Project Timeline}

The project timeline is synchronised to the milestones detailed by
the University, however it is hoped milestones will be delivered early
so that the project can be more rapidly progressed towards the manufacturing
and testing phase where the greatest time constraints lie - it has
been advised that available time in the engine test cells is limited.

\appendix

\section{\label{sec:The-CADMID-Cycle}The CADMID Cycle}

The CADMID cycle is broken down into six stages:
\begin{itemize}
\item Concept

\begin{itemize}
\item In the Concept phase the need for a project is identified.
\item Options for possible solutions to the problem are identified.
\item In this project's context, the literature review is contained here,
primarily.
\end{itemize}
\item Assessment

\begin{itemize}
\item In the Assessment phase the requirements are defined.
\item A single solution to the problem is identified.
\item In this project, the final geometry will be defined here, along with
a broad description of the control system's architecture.
\end{itemize}
\item Demonstration

\begin{itemize}
\item In the Demonstration phase the solution to the requirements is defined
more fully, and is used to demonstrate the compliance of the final
solution with the requirements.
\item In this project full CFD simulation will be used here to demonstrate
efficiency, and the miscellaneous components of the control system
will be defined and included.
\end{itemize}
\item Manufacture

\begin{itemize}
\item In the Manufacturing phase the final solution is built and validated.
\end{itemize}
\item In-service

\begin{itemize}
\item This phase begins when the project's output is accepted into use,
and the project is monitored for obsolescence issues. 
\end{itemize}
\item Disposal

\begin{itemize}
\item In this phase the obsolete solution is removed from use in such a
way as to minimise the impact on the user and the environment.\end{itemize}
\end{itemize}

\end{document}
